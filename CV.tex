\documentclass[]{cv}
\usepackage{color}
\usepackage[utf8]{inputenc}

\usepackage{algolrevived}

\usepackage{ulem}
\usepackage{url}

\usepackage[top=4.5cm,left=1cm,right=1cm,bottom=1cm]{geometry}
\usepackage[usenames,dvipsnames]{xcolor} % Required for custom colours
\usepackage{flowfram}
\usepackage{enumitem} % Required for modifying lists

\setlist{noitemsep,nolistsep} % Remove spacing within and around lists

\renewcommand{\labelitemi}{\diamond}

%\renewcommand{\familydefault}{\ttdefault}
%\setlength{\columnsep}{\baselineskip} % Set the spacing between columns


\newlength{\LeftMainSep}
\setlength{\LeftMainSep}{0.2\textwidth}
\addtolength{\LeftMainSep}{1\columnsep}

\userinfo{
  \begin{flushleft}
    \begin{flushright}
    \Large{\color{black}\textbf{Contato}} \\
    \end{flushright}

    \vspace{1em}

    \large
    Campinas/SP\\
    (19) 988-247-263 % Your phone number

    \vspace{2.5em}

    \begin{flushright}
    \Large{\color{black}\textbf{Email}}
    \end{flushright}

    \large
    \begin{flushright}
      gustavomsvalente
      \normalsize
      @gmail.com 
    \end{flushright}


    \vspace{1em}

    \begin{flushright}
      \Large{\color{black}\textbf{{github.com/}}}
      \large
      gmsvalente
    \end{flushright}
    \normalsize

    %% \vspace{5em}
    %% \url{neural-works.com} % Your URL


    \vspace{8em}
    \begin{flushright}    
    \Large{\color{black}\textbf{{Línguas}}} \\
    \end{flushright}
    \vspace{1em}
    \large
    Inglês avançado\\
    \vfill % Whitespace under this block to push it up under the photo
    \normalsize

  \end{flushleft}
}

\newflowframe{0.2\textwidth}{\textheight}{0pt}{0pt}[left]
\newstaticframe{1.5pt}{\textheight}{\LeftMainSep}{0pt}[line]
\newflowframe{0.65\textwidth}{\textheight}{1.08\LeftMainSep}{0pt}[main]

\begin{staticcontents}{1}
  \tikz{\draw[loosely dotted,color=black,line width=1.2pt,yshift=0](0,0) -- (0,1.05\textheight)}
\end{staticcontents}

\setlength{\parindent}{0px}

\pagestyle{empty} % Disable all page numbering

\begin{document}

\header{gustavo}{VALENTE}{}
\userinfo
\framebreak


\large Formado em Física sou um curioso por ciência, matemática e
tecnologias e tenho paixão por computadores. Tenho habilidade com
linguagens de programação, algoritmos e possuo conhecimentos em TI
como desenvolvimento backend e frontend, um pouco de computação em nuvem e
programação mobile. Tenho facilidade com conceitos modernos de IA,
machine learning e ciência de dados. Possuo também noções de hardware,
\hiphenatino{micro-eletrônica} e microcontroladores. Estou mudando minha área de
atuação para me tornar um \underline{fullstack dev}, procuro empresas de ponta onde possa
utilizar meus conhecimentos e aprender novos.


\section*{\Large{Formação}}

\begin{entrylist}
\Large{
  \entry {2019--------}{Eng. de Controle e Automação [noturno]}{-- Unicamp}\\
  \entry {2009-2015}{\sout{Doutorado Engenharia Elétrica}}{-- (não concluido)}\\
  \entry {2006-2008}{Mestrado Engenharia Elétrica}{-- Unicamp}\\
  \entry {1998-2004}{Graduação Física}{-- Unicamp}
  }
\end{entrylist}


\section*{\Large{Noções + Interesses} \small{(ou softskills WIP)}}

\small{$\bigoplus$ = foco atual}

\vspace{1.0em}

\begin{itemize}[itemsep=7pt]
\normalsize
\item \textbf{OS:} Linux, bash.
\item \textbf{LINGUAGENS DE PROGRAMAÇÃO:} (LISP), $\bigoplus$Clojure, $\bigoplus$Clojurescript, C/C++, Go, Python, Java, Javascript, Fortran, asm.
\item \textbf{BANCO DE DADOS:} Postgresql, $\bigoplus$XTDB, Datomic, Datascript.
\item \textbf{WEB/FRAMEWORKS:} $\bigoplus$html5, $\bigoplus$css, $\bigoplus$React, $\bigoplus$Reagent, $\bigoplus$Re-frame, $\bigoplus$Material-UI.
\item \textbf{MOBILE:} $\bigoplus$Expo, $\bigoplus$React-Native.
\item \textbf{INFRAESTRUTURA TI:} $\bigoplus$Heroku, AWS, Firebase, docker, git.
\item \textbf{ALGORITMOS:} estrutura de dados, algoritmos genéticos, machine learning, redes neurais, IA.
\item \textbf{HARDWARE:} microeletrônica, microcontroladores, PIC, Arduino, RaspberryPi.
\item \textbf{DOCUMENTAÇÃO:} {\large \TeX}.
\end{itemize}

\large

\section*{Objetivos}
Procuro uma empresa inovadora para avançar meus conhecimentos em computação e descobrir novas fronteiras.


\end{document}
